\chapter{Installation}
\label{installation}
% \thispagestyle{empty}

\noindent 
The source code of this R package can be found and downloaded from \url{https://github.com/fpavone/pacs_spline_density} in the branch \textit{ROpenMP}. 
There are different ways of installing it on your machine.

The first one requires the use of \verb|devtools|, a popoular R package. It will also take care of installing all dependencies (Rcpp and RcppEigen) and install the splineDensity package in the same directories of all the other R packages installed.  \\
In R, use the command 
\begin{Verbatim}[commandchars=\\\{\}]
install_github("fpavone/pacs_spline_density",ref="ROpenMP")
\end{Verbatim}
Using \verb|devtools|, it's possible to install the package from the terminal.
Once the source code is downloaded and unzipped, it is enough to run from the package root folder the following commands:
\begin{Verbatim}[commandchars=\\\{\}]
   R -e \textcolor{red}{"library(devtools); install()"} --silent
\end{Verbatim}
To build the documentation in Roxygen, then:
\begin{Verbatim}[commandchars=\\\{\}]
   R -e \textcolor{red}{"library(devtools); document()"} --silent
\end{Verbatim}

This second command will create a subfolder named man where the .Rd files will be stored. This will be useful when calling for "help" from R (e.g., ?smoothSplines).

As an alternative, the second way to complete the installation does not require any additional package, but will throw errors if the packages required for the functioning of splineDensity are not installed.
From the terminal, you need to run:
\begin{verbatim}
   R CMD BUILD <path to folder splineDensity>
   R CMD INSTALL -l <path name of the R library tree> 
      <path name of the package to be installed>
\end{verbatim}

To test the successful installation of the package, it is possible to run one of the examples in the subdirectory tests, either from terminal
\begin{verbatim}
   Rscript <what-else-test.R>
\end{verbatim}
or directly in R.

To build the \verb|Doxygen| documentation of the C++ code instead, you need to navigate to the \textit{src} folder and type 
\begin{verbatim}
   doxygen -g <config-file>
\end{verbatim}
This will create a configuration file (if \verb|<config-file>| is missing, by default it will be named \textit{Doxyfile}), that can be edited to customize the output. \\
To run Doxygen then type:
\begin{verbatim}
   doxygen <config-file>
\end{verbatim}
To get the Reference Manual, then go to the \textit{latex} subfolder that has been generated and type \verb|make|.



%---------------------------------------------------------------

\section{How to enable openMp parallelization}
If your compiler supports openMP parallelization, once you have downloaded the splineDensity package, look for the \textit{Makevars} in the \textit{src} subfolder.\\
Set the \verb|OPENMP| macro defined in it to a non-empty value to activate the parallelization, e.g. \verb|OPENMP = 1|. By default, the option is disabled.

Now, follow the instructions as above to install the package. \\
Once finished, you have the opportunity to exploit the benefits that only a parallel implementation can provide.
