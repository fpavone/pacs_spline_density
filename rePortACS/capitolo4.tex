\chapter{Tutorial}
\label{tutorial}
% \thispagestyle{empty}

% NOTE
EHI WE HAVE TO TEST THESE COMMANDS TO BE SURE THEY ALL WORK!!!

\noindent 
There are different ways for installing the R package splineDensity we created on your machine.

The first one requires the use of devtools, a popoular R package, it will also take care of installing all dependencies and install the splineDensity package in the same directories of all the other R packages installed. 

Once the source code is downloaded from \url{https://github.com/fpavone/pacs_spline_density/tree/ROpenMP} and unzipped, it is enough to run from the package root folder:
\begin{Verbatim}[commandchars=\\\{\}]
   R -e \textcolor{red}{"library(devtools); install()"} --silent
\end{Verbatim}
To build the documentation in Roxygen then:
\begin{Verbatim}[commandchars=\\\{\}]
   R -e \textcolor{red}{"library(devtools); document()"} --silent
\end{Verbatim}

This second command will create a subfolder named man where the .Rd files will be stored. This will be useful when calling for "help" from R (e.g., ?smoothSplines).

As an alternative, to complete the installation does not require any additional package, but will throw errors if the packages required for the functioning of splineDensity are not installed.
From the terminal, run:
\begin{verbatim}
   R CMD BUILD <path to folder splineDensity>
   R CMD INSTALL -l <path name of the R library tree> <path 
     name of the package to be installed>
\end{verbatim}

In a different way, without downloading anything explicitly, simply run:
\begin{Verbatim}[commandchars=\\\{\}]
   R -e \textcolor{red}{"library(devtools);} 
\textcolor{red}{install_github("fpavone/pacs_spline_density",ref="ROpenMP")"} --silent
\end{Verbatim}

To test the successful installation of the package, it is possible to run one of the examples in the subdirectory tests either from terminal
\begin{verbatim}
   Rscript <what-else-test.R>
\end{verbatim}
or loading the script directly in R.

To build the Doxygen documentation of the C++ code instead, navigate to the \textit{src} folder and type 
\begin{verbatim}
   doxygen -g <config-file>
\end{verbatim}
This will create a configuration file (if \verb|<config-file>| is missing, it will be named \textit{Doxyfile}), that can be edited to customize the output. \\
To run Doxygen then type:
\begin{verbatim}
   doxygen <config-file>
\end{verbatim}
To get the Reference Manual, then go to the \textit{latex} subfolder that has been generated and type \verb|make|.



%---------------------------------------------------------------

\section{How to enable openMp parallelization}
If your compiler supports openMP parallelization, once you have downloaded the splineDensity package, look for the \textit{Makevars} in the \textit{src} subfolder.\\
Set the \verb|OPENMP| macro defined in it to a non-empty value to activate the parallelization, e.g. \verb|OPENMP = 2|. By default, the option is disabled.

Now, follow the instructions as above to install the package. \\
Once finished, you have the opportunity to exploit the benefits that only a parallel implementation can provide.
