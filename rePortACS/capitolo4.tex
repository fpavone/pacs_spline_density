\chapter{Installation}
\label{installation}
% \thispagestyle{empty}

\noindent 
The source code of this R package can be found and downloaded from \url{https://github.com/fpavone/pacs_spline_density} in the branch \textit{master}. 
There are different ways of installing it on your machine.

The first one requires the use of \verb|devtools|, a popular R package. It will also take care of installing all dependencies (Rcpp and RcppEigen) and install the splineDensity package in the same directories of all the other R packages installed.  \\
In R, use the commands: 
\begin{Verbatim}[commandchars=\\\{\}]
   library(devtools)
   install_github("fpavone/pacs_spline_density")
\end{Verbatim}
Note that it is required to have the unzip option as \verb|internal|, you can check the unzip setting of your machine with \verb|getOption("unzip")| and possibly change it with \verb|options(unzip = "internal")|.

As an alternative, still using \verb|devtools|, it's possible to install the package from the terminal.
Once the source code is downloaded and unzipped, it is enough to run from the package root folder the following commands:
\begin{Verbatim}[commandchars=\\\{\}]
   R -e \textcolor{red}{"library(devtools); install()"} --silent
\end{Verbatim}
To build the documentation in Roxygen, then:
\begin{Verbatim}[commandchars=\\\{\}]
   R -e \textcolor{red}{"library(devtools); document()"} --silent
\end{Verbatim}
This second command will create a subfolder named man where the .Rd files will be stored. This will be useful when calling for "help" from R (e.g., ?smoothSplines).

A different way to complete the installation does not require any additional package, but will throw errors if the packages required for the functioning of splineDensity are not installed. \\
From the terminal, after downloading the code, you need to run:
\begin{verbatim}
   R CMD INSTALL -l <path name of the R library tree> 
      <path name of the package to be installed>
\end{verbatim}

\medskip

To test the successful installation of the package, it is possible to run the example in the subdirectory tests, either from terminal
\begin{verbatim}
   Rscript test_particle.r
\end{verbatim}
or loading it directly in R. Otherwise the R command \verb|example(smoothSplines)| does the same job.

\medskip

The \verb|Doxygen| documentation of the C++ code can be built from the \textit{src} folder using the provided \textit{Doxyfile}. To run Doxygen type:
\begin{verbatim}
   doxygen Doxyfile
\end{verbatim}
Keep in mind that, in order to create a new documentation with different settings, you need to create a new configuration file, i.e.\textit{Doxyfile}, with:
\begin{verbatim}
   doxygen -g <config-file>
\end{verbatim}
that can be edited to customize the output.

To get the Reference Manual, then go to the \textit{latex} subfolder that has been generated and type \verb|make|.


%---------------------------------------------------------------

\section{How to enable OpenMP parallelization}
If your compiler supports OpenMP parallelization, once you have downloaded the splineDensity package, look for the \textit{Makevars} in the \textit{src} subfolder.\\
Set the \verb|OPENMP| macro defined in it to a non-empty value to activate the parallelization, e.g. \verb|OPENMP = 1|. By default, the option is disabled.

Now, follow the instructions as above to install the package. \\
Once finished, you have the opportunity to exploit the benefits that only a parallel implementation can provide.


%---------------------------------------------------------------

\section*{4.2\quad Installing devtools}
It may be possible to have problems in installing the R package \verb|devtools| in Ubuntu or other Linux distributions.

If the installation throws the following errors:
\begin{verbatim}
ERROR: dependency `curl' is not available for package `httr'
ERROR: dependencies `httr', `memoise' are not available
for package `devtool'...
Warning messages:
1: In install.packages("devtools"):
installation of package `httr' had non-zero exit status
2: In install.packages("devtools"):
installation of package `devtools' had non-zero exit status
\end{verbatim}
it may be solved by installing the following libraries:  \verb|sudo yum -y install| \verb|libcurl libcurl-devel| in CentOS and \verb|sudo apt-get install| \\ \verb|libcurl4-openssl-dev libssl-dev| in Ubuntu.





