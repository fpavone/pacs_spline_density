\chapter{Description of the problem to solve}
\newtheorem*{definition}{Definition}
\label{problem}
% \thispagestyle{empty}

\noindent 
\section{Introduction}
In the analysis of large-scale database systems, information is frequently summarized using a density function, that is, Borel measurable, positive function on a support $\text{I}$ with a unit integral constraint. 

Even though it might seem that density functions are just a special case of functional data, standard FDA methods appear to be inappropriate for their treatment, as they do not consider the particular constrained nature of the data.

This problem is well known in the finite dimensional setting, where specific techniques have been worked out to deal with compositional data, i.e., multivariate data carrying only relative information, usually represented in proportions or percentages. \\
Those techniques are mainly based on a geometric perspective grounded on the Aitchison geometry in the simplex, which properly incorporates the compositional nature of the data (see the 'old but gold' pioneer \citep{aitchinson:bayes}). 

In this context, probability density functions have recently been interpreted as functional compositional data, i.e., functional data carrying only relative information.
To handle this kind of data, the Aitchison geometry has been lately extended to the so called Bayes spaces: a Hilbert space structure for $\sigma$-finite measures, including probability measures, has been worked out, as in \citep{vdboogaart:bayes}.

In order to resort to standard statistical analysis of density functions, a mapping from the Bayes space to the standard $\textit{L}^2$ space is needed. \\
An isometric isomorphism between the Bayes space and the Hilbert space $\textit{L}^2$ of (equivalence classes of) square-integrable real functions on the support $\text{I}$ is defined by the centred log-ratio (clr) transformation, explicitly 
\[	clr[f(x)] = f_c(x) = ln(f(x)) - \frac{1}{\eta}\int_{I} ln(f(x))\, dx 	\]
where $f(x)$ is a density function, $\eta$ is the length of the interval I. \\
The antitranformation is defined as: 
\[  clr^{-1}[f(x)]= \frac{exp(f_c(x))}{\int_{I} f_c(x)\, dx}. 	\]
It's immediate to notice that the following constraint holds:
\[	\int_{I} clr[f(x)]\, dx = 0. 			\]
This additional condition needs to be taken into account for computation and analysis on clr-transformed density functions.

The natural and logical step in smoothing density functions is thus just to express the data in the $\textit{L}^2$ space of clr-transformed densities and to perform the computations there. \\
Consequently, it is guaranteed that all the theoretical properties hold and no additional reasoning is necessary. \\
Nevertheless, the above condition has to be taken into consideration as the smoothing is estimated. 

%%% NOTE (hope to be rigth)
% Remember what dr. Menafoglio told us about the problem: starting from discretized data, we could smooth and then transform them, in order to use them alongside with the unit integral constraint. But it would be uneasy to deal with a constraint on the b-spline coeffients. 
% Therefore, what we do is transform and then smooth in such a way that the constraint is automatically satisfied. It requires a little bit more of calculus but that's a big advantage

% methods for turning raw discrete data into smooth functions
\section{B-spline representation}
The main goal in the preprocessing of discretized distributional observations is finding a smoothed version of histogram data, in other words going from compositional functional data to smooth density functions. 

In order to do this, \citep{paper:pacs} works with the B-spline basis function system. \\
Spline functions are often chosen for the approximation of non-periodic functional data, thanks their fast computation and flexibility. \\
The most famous system of spline for this kind of problem is the smoothing B-spline basis system, developed by de Boor (\cite{ramsay:FDA}). The choice is motivated by the fact that the resulting coefficients of basis functions can be directly used for statistical analysis. 

We recall here the basic facts about B-splines, using the same notation of the reference paper \citep{paper:pacs}.

Let $u,v$ the extremals of the support I, $\Delta\lambda := \lambda_0=u < \lambda_1< \dots < \lambda_g < v = \lambda_{g+1} $ a sequence of knots, and $S_k^{\Delta\lambda}[u,v]$ the vector space of polynomial splines of degree $k$ defined on the interval $[u,v]$ using the knots $\Delta\lambda$. It is known that  $dim(S_k^{\Delta\lambda}[u,v]) = g + k + 1$. \\
A spline in this space has the following form:
\[  s_k(\textbf{x})=\sum\limits_{i=-k}^{g}b_iB_i^{k+1}(\textbf{x}) \]
where ${B_i^{k+1}}$ are B-splines of degree $k$ and form a basis in $S_k^{\Delta\lambda}[u,v]$. \\
For this representation, additional knots are required:
\[  \lambda_{-k}= \dots = \lambda_{-1} = \lambda_0,  \ \ \ \ \lambda_{g+1}= \dots = \lambda_{g+k+1}. \]
%%% NOTE: maybe we can skip the upper part
In matrix notation, the spline can be rewritten as:
\[  s_k(\textbf{x})=  \textbf{C}_{k+1}(\textbf{x})\textbf{b}, \]
where $ \textbf{C}_{k+1}(\textbf{x})$ is the \textit{collocation matrix}, defined as
\[  \textbf{C}_{k+1}(\textbf{x}) =
\begin{bmatrix}
B_{-k}^{k+1}(x_1)  & \dots  & B_{g}^{k+1}(x_1) \\
\vdots & \ddots & \vdots \\
B_{-k}^{k+1}(x_n) &  \dots  & B_{g}^{k+1}(x_n)
\end{bmatrix} \in \mathbb{R}^{n, g+k+1} \]
and  $\textbf{b}= [ b_{-k}, \dots, b_{g}]^T$ is \textit{the vector of B-spline coefficients} of $s_k(\textbf{x})$.\\
Moreover, using properties of B-splines, it is possible to write the derivative of order $l$ of the spline $s_k(\textbf{x})$ as
\[  s_k^{(l)}(\textbf{x})=  \textbf{C}_{k+1-l}(\textbf{x}) \, \textbf{b}^{(l)}, \]
where $\textbf{b}^{(l)}$ can be computed recursively:
\[ \textbf{b}^{(l)} = \textbf{D}_l \textbf{L}_l \textbf{b}^{(l-1)}   =  \textbf{D}_l \textbf{L}_l \dots \textbf{D}_1 \textbf{L}_1\textbf{b} = \textbf{S}_l\textbf{b}, \]
\[ \textbf{D}_j =  (k+1-j) \, \text{diag}(d_{-k+j},\dots, d_g)    \]
\[ \text{with} \ d_i = \frac{1}{\lambda_{i+k+1-j}-\lambda_i} \qquad \forall i = -k+j,\dots, g,  \]
\[ \textbf{L}_j =  \begin{bmatrix}
-1  & 1 &  &\\
 & \ddots & \ddots&\\
 & & -1&1
\end{bmatrix}  \in \mathbb{R}^{g+k+1-j,g+k+2-j}.\]

\section{The optimal smoothing problem} \label{optimal}
Our goal is to find a function $f(x)$ that is a smooth approximation, close enough to the given data points $(x_i,y_i)$. Hence $f(x)$ has to fulfil the following optimization problem:
\begin{equation*} 
\begin{aligned}
& \underset{f}{\text{minimize}}
& & \int_{x_1}^{x_n} [f^{(l)}(x)]^2 \,dx \\
& \text{subject to}
& & \sum\limits_{i=1}^{n} \, [w_i(y_i-f(x_i))]^2 \le S
\end{aligned}
\end{equation*}
where the objective function is a measure of non-smoothness of $f(x)$ and the constraint is a measure of the closeness of fit that takes into account data accuracy using weights. \\
The solution is known to be a natural spline $s_k(\textbf{x})$ of degree $k=2l-1$.
It follows that, for $l \ge 2$,  
\[s_k^{(l+j)}(x_1)=s_k^{(l+j)}(x_n)=0, \ \ \ j=0,1,\dots,l-2.\]
In order to find $\textbf{b}$, we can plug this result into the dual problem:
\begin{equation*} 
\begin{aligned}
& \underset{f}{\text{minimize}}
& &  J_l(f) \\
\end{aligned}
\end{equation*}
\[\text{where} \qquad J_l(f) :=\int_{x_1}^{x_n} [f^{(l)}(x)]^2 \,dx + \alpha \sum\limits_{i=1}^{n}[w_i(y_i-f(x_i))]^2\]

Given data $(x_i,y_i), \ u\le x_i \le v$, stored as $\textbf{x} = [x_1, \dots, x_n]^\top$, $\textbf{y} = [y_1, \dots, y_n]^\top$, let $w_i \ge 0, \ i=1, \dots, n$ be the weights related to each observation, $\Delta\lambda$ the sequence of knots, $n \ge g+1 $ and given $\alpha \in (0,1)$, the functional $J_l(f)$ can be rewritten using matrix notation:
\[J_l(\textbf{b}) = \textbf{b}^\top \textbf{N}_{kl}\textbf{b} + \alpha [\textbf{y}-\textbf{C}_{k+1}(\textbf{x})\textbf{b}]^\top \textbf{W} [\textbf{y}-\textbf{C}_{k+1}(\textbf{x})\textbf{b}]\]
where $\textbf{N}_{kl} =  \textbf{S}_l^\top \textbf{M}_{kl} \, \textbf{S}_l$ is positive semidefinite, 
\[  \textbf{M}_{kl} =
\begin{bmatrix}
(B_{-k+l}^{k+1-l}, B_{-k+l}^{k+1-l}) & \dots  & (B_{g}^{k+1-l}, B_{-k+l}^{k+1-l}) \\
\vdots & \ddots & \vdots \\
(B_{-k+l}^{k+1-l}, B_{g}^{k+1-l}) &  \dots  & (B_{g}^{k+1-l}, B_{g}^{k+1-l})
\end{bmatrix} \in \mathbb{R}^{g+k+1-l, g+k+1-l} \]
and \[(B_{i}^{k+1-l}, B_{j}^{k+1-l}) = \int_u^v B_{i}^{k+1-l}(x)B_{j}^{k+1-l}(x) \, dx.\]
$\textbf{M}_{kl}$ is positive definite, because each element is the scalar product in $L^2([u,v])$ of basis functions. \\
Since we are working with smoothed clr-transformed density functions, we have to add the condition:
\[\int_u^v s_k(x) \, dx = 0.\] 
From B-spline properties we know that 
\[  s_k(\textbf{x})=\sum\limits_{i=-k}^{g}b_iB_i^{k+1}(\textbf{x}) \]
is the derivative of the spline
\[  s_{k+1}(\textbf{x})=\sum\limits_{i=-k-1}^{g}c_iB_i^{k+2}(\textbf{x}) \]
if
\[b_i = (k+1)\frac{c_i-c_{i-1}}{\lambda_{i+k+1}-\lambda_{i}}\ \ \  \forall i = -k, \dots, g.\]
Hence 
\[ 0 = \int_u^v s_k(x) \, dx = [s_{k+1}(x)]_u^v = s_{k+1}(\lambda_{g+1})-s_{k+1}(\lambda_{0}) = c_g-c_{-k-1},\]
therefore $c_g = c_{-k-1}$.\\
Finally, we have identified a relationship between $\textbf{b}$ and $\textbf{c}$:
\[\textbf{b} = \textbf{D}\textbf{K}\textbf{c},\] 
where $ \textbf{D} = \textbf{D}_0$ and
\[  \textbf{K} =
\begin{bmatrix}
1 &0&0& \dots  & -1 \\
-1 &1&0& \dots  & 0 \\
0 &-1&1& \dots  & 0 \\
\vdots &\vdots & \ddots & \ddots& \vdots \\
0 &0 & \dots&-1  & 1
\end{bmatrix} \in \mathbb{R}^{g+k+1, g+k+1}. \]
Hence the functional to minimize can be rewritten depending on $\textbf{c}$:
\[J_l(\textbf{c}) = (\textbf{D}\textbf{K}\textbf{c})^\top \textbf{N}_{kl}\textbf{D}\textbf{K}\textbf{c} + \alpha [\textbf{y}-\textbf{C}_{k+1}(\textbf{x})\textbf{D}\textbf{K}\textbf{c}]^\top \textbf{W} [\textbf{y}-\textbf{C}_{k+1}(\textbf{x})\textbf{D}\textbf{K}\textbf{c}]\]
In order to find the minimum, we set to 0 the derivative of $J_l(\textbf{c})$ w.r.t. $\textbf{c}$ and we obtain:
\[ [\alpha^{-1}(\textbf{D}\textbf{K})^\top \textbf{N}_{kl} \textbf{D}\textbf{K}+ (\textbf{C}_{k+1}(\textbf{x})\textbf{D}\textbf{K})^\top\textbf{W}\textbf{C}_{k+1}(\textbf{x})\textbf{D}\textbf{K}]\textbf{c} = (\textbf{C}_{k+1}(\textbf{x})\textbf{D}\textbf{K})^\top \textbf{W}\textbf{y}.\]
If $\alpha^{-1}(\textbf{D}\textbf{K})^\top \textbf{N}_{kl} \textbf{D}\textbf{K}+ (\textbf{C}_{k+1}(\textbf{x})\textbf{D}\textbf{K})^\top\textbf{W}\textbf{C}_{k+1}(\textbf{x})\textbf{D}\textbf{K}$ is invertible, then there exists a unique solution $\textbf{c}^\star$:
\[\textbf{c}^\star = [\alpha^{-1}(\textbf{D}\textbf{K})^\top \textbf{N}_{kl} \textbf{D}\textbf{K}+ (\textbf{C}_{k+1}(\textbf{x})\textbf{D}\textbf{K})^\top\textbf{W}\textbf{C}_{k+1}(\textbf{x})\textbf{D}\textbf{K}]^{-1}\textbf{K}^\top\textbf{D}^\top\textbf{C}_{k+1}^\top \textbf{W}\textbf{y},\]
otherwise we can find a minimum norm solution
\[\textbf{c}^\star = [\alpha^{-1}(\textbf{D}\textbf{K})^\top \textbf{N}_{kl} \textbf{D}\textbf{K}+ (\textbf{C}_{k+1}(\textbf{x})\textbf{D}\textbf{K})^\top\textbf{W}\textbf{C}_{k+1}(\textbf{x})\textbf{D}\textbf{K}]_{m}^{-}\textbf{K}^\top\textbf{D}^\top\textbf{C}_{k+1}^\top \textbf{W}\textbf{y}.\]
In both cases the vector of B-spline coefficients $\textbf{b}^\star$ is obtained by:
\[\textbf{b}^\star = \textbf{D}\textbf{K}\textbf{c}^\star.\] 


