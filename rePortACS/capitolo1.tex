\chapter{}
\newtheorem*{definition}{Definition}
\label{Introduzione}
% \thispagestyle{empty}
\noindent 
\section{Introduction to the problem}
Analysing large-scale database systems, one of the first things to do is to summarize information in a density function, that is, Borel measurable, positive functions on a support I with a unit integral constraint. In this way, the intrinsic variability in the data is preserved and statistical analysis can be done according to the already existing methodologies. \\
$\text{L}^2$ metric is not appropriate to tackle this problem because of the unit-integral constraint and because density functions take into account relative contributions of Borel sets on real line to the overall probability on the support of the variables. \\
On the contrary, Bayes spaces with separable Hilbert space properties seem to fit perfectly. 
Hence the general idea to tackle this problem is to to express the data in the $\text{L}^2$ space  and to perform the computations there. For this reason a mapping  from the Bayes space to the standard $\text{L}^2$ space is needed and it follows from the property that the logarithm of probability density functions is square-integrable (**[11]**): considering the finite interval support case I, where Lebesgue measure is used , the simplest isomorphism is the centred logratio (clr) transformation, defined as 
\begin{center}
	\begin{math}
	clr[f(x)]= f_c(x) ln(f(x)) - \frac{1}{\eta}\int_{I} ln(f(x))\, dx
	\end{math}
\end{center}
where $f(x)$ is a density function, $\eta$ is the length of the interval I.
The antitranformation is defined as: 
\begin{center}
	\begin{math}
	clr^{-1}[f(x)]= \frac{exp(f_c(x))}{\int_{I} f_c(x)\, dx}.
	\end{math}
\end{center}
 The following condition has to be considered in the analysis:
 \begin{center}
 	\begin{math}
 	\int_{I} clr[f(x)]\, dx = 0.
 	\end{math}
 \end{center}


\section{Why B-spline }
In practice the aggregation of individual observations to a discretized form of histogram data is done and a smoothed version of the distribution has to be found.
Spline functions are often choosed for the approximation of non-periodic functional data since they allow fast computation and flexibility. The most famous system of spline for this kind of problem is the smoothing B-spline basis system developed by de Boor. This is because the resulting coefficients of basis functions can be directly used for statistical analysis. %2005_Book_FunctionalDataAnalysis