\chapter{}
\newtheorem*{definition}{Definition}
\label{Introduzione}
% \thispagestyle{empty}
\noindent 
\section{Introduction to the problem}
Analysing large-scale database systems, one of the first things to do is to summarize information in a density function, that is, Borel measurable, positive functions on a support I with a unit integral constraint. In this way, the intrinsic variability in the data is preserved and statistical analysis can be done according to the already existing methodologies. \\
$\text{L}^2$ metric is not appropriate to tackle this problem because of the unit-integral constraint and because density functions take into account relative contributions of Borel sets on real line to the overall probability on the support of the variables. \\
On the contrary, Bayes spaces with separable Hilbert space properties seem to fit perfectly. 
A mapping between this two spaces follows from the property that the logarithm of probability density functions is square-integrable (**[11]**): considering the finite interval support case I, where Lebesgue measure is used , the simplest isomorphism from the Bayes space to the standard $\text{L}^2$ space is  the centred logratio (clr) transformation, defined as 
\begin{center}
	\begin{math}
	clr[f(x)]= f_c(x) ln(f(x)) - \frac{1}{\eta}\int_{I} ln(f(x))\, dx
	\end{math}
\end{center}
where $f(x)$ is a density function, $\eta$ is the length of the interval I.
The antitranformation is defined as: 
\begin{center}
	\begin{math}
	clr^{-1}[f(x)]= \frac{exp(f_c(x))}{\int_{I} f_c(x)\, dx}.
	\end{math}
\end{center}
 The following condition has to be considered in the analysis:
 \begin{center}
 	\begin{math}
 	\int_{I} clr[f(x)]\, dx = 0.
 	\end{math}
 \end{center}
%However, in practice the aggregation of individual observations to a discretized form of histogram data is done and a smoothed version of the distribution has to be found.

\section{Why B-spline }

See p.119 J. Ramsay and B.W. Silverman, Functional Data Analysis, 2nd ed., Springer, New York, 2005